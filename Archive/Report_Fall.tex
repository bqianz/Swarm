\documentclass{article}
\usepackage[utf8]{inputenc}
\usepackage{amssymb,amsmath,amsthm,mathtools}
\usepackage{subcaption}
\usepackage{graphicx}
% \usepackage[margin=0.9in]{geometry}

\newcommand{\p}{\partial}
\newcommand{\al}{\vec{\alpha}}
\newcommand{\h}{\theta}
\newcommand{\g}{\gamma}
\newcommand{\hi}{\theta_i}
\newcommand{\hj}{\theta_j}
\newcommand{\ei}{\vec{\mathbf{e}}_1}
\newcommand{\ej}{\vec{\mathbf{e}}_2}
\newcommand{\ek}{\vec{\mathbf{e}}_3}
\newcommand{\er}{\vec{\mathbf{e}}_R}
\newcommand{\eh}{\vec{\mathbf{e}}_\theta}
\newcommand{\ep}{\vec{\mathbf{e}}_\phi}
\newcommand{\ehi}{\vec{\mathbf{e}}_{\theta_i}}
\newcommand{\epi}{\vec{\mathbf{e}}_{\phi_i}}
\newcommand{\X}{\vec{\mathbf{X}}}
\newcommand{\R}{\mathbb{R}}
\def\*#1{\mathbf{#1}}
% \newcommand{\df}{\vcentcolon=}
\newcommand{\kp}{\kappa}
\newcommand{\norm}[1]{\left\lVert#1\right\rVert}
\newcommand{\eq}[1]{\begin{align}#1\end{align}}
\setlength{\jot}{10pt} % row spacing in align

\title{Hyperbolic Sphere} % Poster title
\date{October 2017}
\author{} % Author(s)


\begin{document}
\maketitle
\section{The Question}

The integro-differential equation on the hyperbolic sphere:
\eq{
\rho_t + \nabla_H \cdot (\rho v) =  0, \\
v = -\nabla_H K * \rho, \label{velocity}
}

\section{Set-up on the unit Hyperbolic Sphere}
\subsection{Hyperbolic Coordinates}
We use the parametrization $\mathbf{H}^2$ from \cite{kimura}, as visualized on the upper sheet hyperboloid.Let $\X = x\ei + y\ej + z\ek$, where $\ei,\ej,\ek$ are the standard cartesian basis and
\begin{align*}
x &= \cos{\phi} \sinh{\h}, \\
y &= \sin{\phi} \sinh{\h}, \\
z &= \cosh{\h}.
\end{align*}
Hyperbolic distance $\tilde{\h}$ as defined by the hyperbolic inner product is:
\[
\cosh{\tilde{\h}} =  \cosh{\h_i}\cosh{\h_j} - \sinh{\h_i}\sinh{\h_j}\cos{(\phi_i - \phi_j)},
\]
Note that hyperbolic distance is not proportional to the arclength visually.

Define the hyperbolic basis in $\R^3$ as the following:
\begin{align*}
\er &= \frac{\X}{\norm{\X}_H} \\
\eh &= \frac{\frac{\p \X}{\p \h}}{\norm{\frac{\p \X}{\p \h}}_H } = \cos{\phi}\cosh{\h} \ei + \sin{\phi}\cosh{\h} \ej + \sinh{\h}\ek \\
\ep &= \frac{\frac{\p \X}{\p \phi}}{\norm{\frac{\p \X}{\p \phi}}_H } = -\sin{\phi}\ei + \cos{\phi}\ej
\end{align*}



\subsection{Operators on $\mathbf{H}^2$}
\begin{align*}
\nabla_H f &= \frac{\p f}{\p \h} \eh + \frac{1}{\sin{\h}} \frac{\p f}{\p \phi} \ep \\
\nabla_H \cdot F &=\frac{1}{\sinh{\h}} \left(\cosh{\h} F_\h + \sinh{\h}\frac{\p}{\p\h} F_\h  \right) + \frac{1}{\sinh{\h}} \frac{\p}{\p \phi}F_\phi \\
\Delta_H f &= \nabla_H \cdot \nabla_H f \\
&= \coth{\h} \frac{\p f}{\p \h} + \frac{\p^2 f}{\p \h^2} + \frac{1}{\sinh^2{\h}} \frac{\p^2 f}{\p \phi^2} 
\end{align*}
For an interaction potential that takes two particle positions as arguments, i.e $K(\hi,\phi_i,\hj,\phi_j)$, we define its gradient with respect to the i-th particle as
\[
\nabla_i K = \frac{\p f}{\p \hi} \ehi + \frac{1}{\sin{\hi}} \frac{\p f}{\p \phi_i} \epi.
\]

Convolution centered at $\X_i$ is defined
\[
K * \rho ((\hi,\phi_i) = \int_{0}^{2\pi} \int_{0}^{\infty} K(\hi, \phi_i, \hj, \phi_j) \rho(\hj,\phi_j) \sinh{\hj} d\hj d\phi_j.
\]

\section{Choice of Potential on $\mathbf{H}^2$}

The function satisfying $\Delta_H G = -\delta$ according to \cite{kimura} is
\[
G( \tilde{\h} ) = - \frac{1}{2\pi} \log{ \tanh{\frac{\tilde{\h}}{2}}}.
\]
We want to add to $G$ an attraction term $A$. One could naively solve the equation $\Delta_H A(\tilde{\h}) = C$ for $\h \neq 0$ and $C$ some positive constant to find a family of satisfying functions
\[
A_\gamma(\tilde{\h}) = \frac{\gamma}{2\pi} \log{ \sinh{ \tilde{\h}}}.
\]
We can verify that in fact,
\[
\Delta_H A_\gamma = \gamma \left( \delta + \frac{1}{2\pi} \right)
\]
Then for any $\gamma \neq 1$, the potential
\[
\setlength{\fboxsep}{3\fboxsep}\boxed{
K(\tilde{\h}) = -\frac{1}{2\pi} \log{ \tanh{ \frac{\tilde{\h}}{2}}} + \frac{\gamma}{2\pi} \log{ \sinh{ \tilde{\h}}}
}
\]
satsifies 
\[
\Delta_H K = (1-\gamma)\left( -\delta + \frac{\gamma}{2\pi(1-\gamma)} \right).
\]
By the density ODE following a particle path:
\begin{align}
\frac{D}{Dt} \rho
&=-\rho (-\Delta_H K * \rho),\label{approach}\\
&=-\rho (1-\gamma) \left( \rho - \frac{\gamma M}{2\pi (1-\gamma)}\right).
\end{align}
For $\rho = \frac{\gamma M}{2\pi (1-\gamma)}$ to be a stable equilibrium of the ODE, we must choose $\gamma \in (0,1)$.


\begin{thebibliography}{3}
\bibitem{swarm} 
R. C. Fetecau, Y. Huang and T. Kolokolnikov, Swarm dynamics and equilibria for a nonlocal aggregation model , Nonlinearity, Vol. 24, No. 10, 2681-2716 (2011)
 
\bibitem{kimura}
Yoshifumi Kimura, Vortex motion on surfaces with constant curvature, Proc. R. Soc. Lond. A (1999) 455, 245-259

\bibitem{howard}
Cohl, Howard. (2011). Fundamental Solution of Laplace's Equation in Hyperspherical Geometry. Symmetry, Integrability and Geometry: Methods and Applications. 7. . 10.3842/SIGMA.2011.108.

\end{thebibliography}


\end{document}